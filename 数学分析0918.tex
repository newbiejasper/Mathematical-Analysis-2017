\documentclass{ctexart}
\usepackage{amsfonts} %花写数学字体
\usepackage{enumerate}
\usepackage{amsmath}  %定理,引理等
\newtheorem{theorem}{定理}  
\newtheorem{myDef}{定义} 
\newtheorem{lemma}{Lemma}  
\newtheorem{proof}{Proof}[section]  
\setlength\parindent{0em} %开头不缩进,如果需要缩进,单独设置 \setlength\parindent{2em}


\title{数学分析习题课1}
\author{高家兴}
\date{2017.09.18}

\begin{document}
\maketitle

\section{集合有界}
对任意集合$E\subset \mathbb{R}, E \neq \emptyset$,

\begin{enumerate}[(1)]
	\item 有上界:$\exists M\in\mathbb{R}$,使得对$\forall x\in E, x\leq M.$
	\item 有下界:$\exists m\in\mathbb{R}$,使得对$\forall x\in E, x\geq m.$
	\item 有界:$\exists M>0$,使得对$\forall x\in E, |x|\leq M.$
	\item 无界:$\forall M>0$,$\exists x\in E, |x|\ge M.$
\end{enumerate}

这里我们需要熟练掌握的是有界和无界的数学表述,接下来我们还会遇到函数有界和无界的概念。我们还要学会$\exists$和$\forall$的表达方式,可以对一些数学表述,作出相应的否定表达,这一点在数学分析中用的相当多,尤其是在反证法中。

\section{确界}
对于确界的理解,有几点需要注意的地方。
\begin{enumerate}[(1)]
	\item 记住上(下)确界的$\epsilon$语言表述。(课本第6页定义1.1.2)
	\item 上确界是最小上界,下确界是最大下界。
	\item $-\infty,\infty \notin \mathbb{R}$。
	\item 数集的上(下)确界如果存在,则必定唯一。
\end{enumerate}

\begin{theorem}
(确界存在定理)非空有上界的实数集必有上确界;非空有下界的实数集必有下确界。
\end{theorem}

注:对于确界存在定理,注意里面强调的实数集。在其他数集中不一定成立,例如:

在有理数中,$a_1 = 1, a_2 = 1.4, a_3 = 1.41,\cdots$,即$a_n$是$\sqrt{2}$截取前$n$位有效数字,显然$\{a_n:i = 1,2,\cdots\}$单调递增,且上确界是$\sqrt{2},不是有理数,因此有上界的有理数集不一定有上界。$

\section{一些常用不等式}
\begin{enumerate}[(1)]
	\item $||x|-|y||\leq|x-y|\leq|x|+|y|$
	\item (Bernoulli不等式)对$\forall n,x\geq-1,(1+x)^n\geq1+nx$
\end{enumerate}

\section{函数}
\begin{enumerate}[(1)]
	\item 函数的定义三要素:定义域,值域,对应法则。\par 有一些数学课程中常用的函数,他们常常被用来举反例,或者经过变化之后造出反例。我们需要记住的是Dirichlet函数,Gauss取整函数,特征函数(示性函数)。Riemann函数了解即可。
	\item 函数的运算
	\begin{itemize}
		\item 在共同定义域中进行四则运算
		\item 限制和延拓
		\item 函数的复合:内函数的值域需要包含在外函数的定义域中,一般不满足交换律
	\end{itemize}
	\item 反函数
	\begin{myDef}
	设$f:X\rightarrow Y$是一个一一对应。定义函数$g:Y\rightarrow X$如下:对任意$y\in Y$,函数值$g(y)$规定为由关系式$y = f(x)$所唯一确定的$x\in X$. 这样定义的函数$g(y)$称为是函数$f(x)$的反函数,记为$g = f^{-1}$
	\end{myDef}
	性质:
	\begin{equation*}
	\begin{aligned}
	f(f^{-1}(y))&=y\\
	f^{-1}(f(x))&=x
	\end{aligned}
	\end{equation*}
	

	\item 函数的有界性\par
	设$y = f(x)$是定义在$X$上的函数,
	\begin{itemize}
		\item 若$\exists$常数$M$,使得对$\forall x\in X$,都有$f(x)\leq M$,则称$f(x)$在$X$上有上界
		\item 若$\exists$常数$m$,使得对$\forall x\in X$,都有$f(x)\geq m$,则称$f(x)$在$X$上有下界
		\item 若$\exists$常数$M>0$,使得对$\forall x\in X$,都有$|f(x)|\leq M$,则称$f(x)$在$X$上有界
	\end{itemize}

	\item 函数的单调性,周期性和奇偶性

\section{习题}
\begin{enumerate}[(1)]
	\item 需要看的例题:例1.2.6,例1.3.2,例1.3.7,例1.3.9
	\item 课后习题:15,16,26,27
\end{enumerate}

\end{enumerate}



























\end{document}